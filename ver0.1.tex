\documentclass[a4paper,11pt]{article}
\usepackage[T1]{fontenc}				% Para que se puedan formar tildes como una sola letra
\usepackage[utf8]{inputenc}             % Para poder escribir tildes
\usepackage{textcomp}					% M\`as caracteres acentuados
\usepackage{multicol}					% Crea ambientes donde se puede escribir en varias columnas
\usepackage{soul}
\usepackage{mathtools}
\usepackage{amsmath,amssymb,amsfonts,latexsym,cancel}	% Paquete para la creaci\'on de ambientes y s\'imbolos matem\'aticos
\usepackage{amssymb}
\usepackage{multirow}
\usepackage[table,xcdraw,dvipsnames]{xcolor}
\usepackage{anysize}
\usepackage{fancyhdr}
\usepackage{xcolor}	
\usepackage{graphicx}	
\usepackage{caption}
\usepackage{adjustbox}
\usepackage[colorlinks=true,backref=true,linkcolor=azul,citecolor=g,urlcolor=azul]{hyperref}
\IfFileExists{enumitem.sty}{\usepackage{enumitem}}{}
\usepackage{floatrow}
\usepackage{setspace}
\title{Algebra I - Resueltos 2021}

%---------------------------------------------------------------------------------------------------------------------------
					% Para ajustar m\'argenes del documento
\marginsize{1cm}{1cm}{0.5cm}{0.3cm}
%\marginsize{L}{R}{U}{D}
\captionsetup{font={scriptsize}}
%\setlength{\captionmargin}{20pt}
%\newcommand{\captionfonts}{\scriptsize}
%\makeatletter  % Allow the use of @ in command names
%\long\def\@makecaption#1#2{%
%  \vskip\abovecaptionskip
%  \sbox\@tempboxa{{\captionfonts #1: #2}}%
%  \ifdim \wd\@tempboxa >\hsize
%    {\captionfonts #1: #2\par}
%  \else
%    \hbox to\hsize{\hfil\box\@tempboxa\hfil}%
%  \fi
 % \vskip\belowcaptionskip}
%\makeatother

\onehalfspacing

\begin{document}
\makeatletter
\renewcommand{\maketitle}{
\begin{center}
\begin{normalsize}\textbf{\@title}\end{normalsize}\\
\begin{normalsize}\@author\end{normalsize}
\end{center}
}
\title{Álgebra I - Ejercicios Resueltos - 2do Cuatrimestre 2021} 
\author{santi}
\maketitle

Esto es una beta con los ejercicios resueltos de Algebra hasta el primer parcial. Acordate que esta hecho por alumnos para alumnos asi que puede contener \textbf{errores}, así como también errores. Lo voy a terminar post 15 de Diciembre cuando me libere con los parciales. 

Muchas gracias a Teresa, Nico, Georgi, Vicky, Juli y Sergio por hacer esto posible y a mis compañeros Tew, Lucho, Nico, Ivo y Marian[etc]
%\chapter{Conjuntos, Relaciones y Funciones}

\section{Práctica 1 - Conjuntos, Relaciones y Funciones}
\subsection{Conjuntos}
1. i) Verdadero. ii) Verdadero. iii) Verdadero. iv) Falso. v) Falso.\\
2. i) Falso pues \(3\not\in A\). ii) Verdadero. iii) Falso pues \(\lbrace\lbrace3\rbrace\rbrace\subset A\).\\
 iv) Verdadero. v) Verdadero. vi) Verdadero. vii) Verdadero \\
 viii) Falso pues \(3\notin A\). ix) Falso pues \(\emptyset \not\in A\) \\
 x) Verdadero. xi) Falso pues \(A \not\in A\) xii) Verdadero. \\
3.
i) Como B tiene a los elementos de A uno a uno (recordemos que no importa el orden) entonces \(A \subseteq B\)\\
ii) \(A \not \subseteq B\), pues \(\not\exists x \in A\) tal que \(x \not\in B (i.e. ./x = 3)\)\\
iii) \(A \not \subseteq B\), pues \(\sqrt[]{x^2}<\sqrt[]{3}\leftrightarrow |x|<\sqrt[]{3} \) y como \(\sqrt[]{3}<3 \ luego \exists x\in A / x \not\in B\)\\
iv) \(A \not \subseteq B\)\\
4. \[
A=\{1, -2, 7, 3\} , B=\{1, \{3\}, 10\} C=\{-2, \{1,2,3\}, 3\}
\]
\[i) A \cap (B \triangle C)\]
\[(B \triangle C) = \{1,\{3\}, 10, -2, \{1,2,3\},3\} \:y\: A=\{1, -2, 7, 3\}\:entonces: \]
\[A \cap (B \triangle C) = \{1, -2, 3\}\]
\[ii) (A \cap B) \triangle (A \cap C)\]
\[(A \cap B)=\{1\}\:y\:(A \cap C)=\{-2,3\}\]
\[(A \cap B) \triangle (A \cap C)=\{1,-2,3\}\]
\[iii) A^{c}\cap B^{c}\cap C^{c}\:\:bueno \:hacelo\: vos.\]
asdasdasd\\
5. Dados \(A\;B\;C\) subconjuntos de un conjunto ref V, describir \((A\cap B\cap C)^{c}\) en terminos de\\
\textit{Intersecciones y complentos:}\\
\(((A\cap B)\cap C)^{c}\:\:\:\:\:\:\:\:Por\:asociatividad\: de\: la\: union\)\\
\((A\cap B)^{c}\cap C^{c}\:\:\:\:\:\:\:\:\:De\: Morgan\)\\
\(A^{c}\cap B^{c}\cap C^{c}\:\:\:\:\:\:\:\:Idem\)\\
\textit{Uniones y complementos:}\\
\(((A\cap B)\cap C)^{c}\:\:\:\:\:\:\:\:Por\:asociatividad\: de\: la\: union\)\\
\((A\cap B)^{c}\cup C^{c}\:\:\:\:\:\:\:\:\:De\: Morgan\)\\
\(A^{c}\cup B^{c}\cup C^{c}\:\:\:\:\:\:\:\:Idem\)\\
6. Hecho en dibujitos que hay que subir (y me da pajulis)\\
7. i) \((A\cap B^{c})\cup((B\cap C) \cup A^{c})\)\\
ii) \((A\triangle C)\cap B^{c}\)\\
iii) \((A\cap B)\triangle(B\cap C)\cup(A\cap C)\cap B^{c}\)\\
8. i) \(A={1},\:\:\:\mathcal{P}(A)=\{\{1\},\emptyset\}\)\\
ii)\(A=\{a,b\},\:\:\:\mathcal{P}(A)=\{\{a\},\{b\},\{a,b\},\emptyset\}\)\\
iii)\\
9. Probar etc. \\
Bueno, qué corno hago? En criollo, tenemos que probar que partes de A es un subconjunto de partes de B \textbf{\textit{sí sólo sí}} A está contenido en B. Para probar este tipo de implicaciones, tenemos que probar la ida (\(\Rightarrow\)) y luego la vuelta (\(\Leftarrow\))\\
Recuerdo: \(\mathcal{P}(A)=\{X:X\subseteq A\}\:\: o\:\:tambien\:\: X \in \mathcal{P}(A)\iff X\subseteq A\)\\
Ida (\(\Rightarrow\)): Asumo que \(\mathcal{P}(A)\subseteq \mathcal{P}(B)\) quiero ver que \(A\subseteq B \)
Tomo \(x \in A \) quiero ver que \(x \in B \). Se que \(A \subseteq A\) . A, el conjunto, es un elemento de \(\mathcal{P}(A)\) y se escribe \(A\in \mathcal{P}(A)\) y como se que \(\mathcal{P}(A)\subseteq \mathcal{P}(B)\), entonces por transitividad,  \(A\in \mathcal{P}(B)\). Pero, ¿qué significa que \(A\in \mathcal{P}(B)\)? Por definición de \(\mathcal{P}(B)\), A tiene que ser un subconjunto de B. Y como es un subconjunto,  \(A\subseteq B \).\\
Vuelta (\(\Leftarrow\)): Sabiendo que \(A\subseteq B \) q.v.q. \(\mathcal{P}(A)\subseteq \mathcal{P}(B)\) \\
Tomo \(X \in \mathcal{P}(A) \) entonces \(X \subseteq A \) y entonces \(X \subseteq A \subseteq B\) y por transitividad, \(X \subseteq B\). A la vez, por definici'on, \(X \subseteq B \subseteq \mathcal{P}(B) \), luego \(X \subseteq {P}(B) \), por lo tanto \(X \subseteq \underline{\mathcal{P}(A) \subseteq \mathcal{P}(B)} \) que es lo que queriamos probar.\\
10. Tablas de verdad\\
11. Tablas de verdad \\
12. Tablas de verdad (Agregar notas sobre cuantificadores) \\
13. Tablas de verdad \\
14. Similar \\
15. Sean \(A=\{1,2,3\} B=\{1,3,5,7\}\). Hallar \(A\times A, A\times B, (A\cap B)\times(A\cup B)\)\\
\(A \times A= \{(1,1),(1,2),(1,3),(2,1),(2,2),(2,3),(3,1),(3,2),(3,3)\} \)\\
\(A \times B= \{(1,1),(1,3),(1,5),(1,7),(2,1),(2,2),(2,5),(2,7),(3,1),(3,3),(3,5),(3,7)\} \)\\
\((A\cap B)\times(A\cup B)=\{(1,1),(1,3),(1,5),(1,7),(3,1),(3,3),(3,5),(3,7)\} \)\\
16. Probalo vos. \\
\subsection{Relaciones}
17. i. Es relacion? Si, pues todos los elementos que se relacionan de A en B existen en A o B segun especifica la relacion.\\
ii. No es, ya que \( (3,2) \in R\) pero \(2 \not\in B\)\\
iii. Si, idem i.\\
iv. Si, idem i. \\
18. i) \(\{(a,b)\in A \times B\: : \: a\leq b\}\)\\
ii) \(\{(a,b)\in A \times B\: : \: a > b\}\)\\
iii) \(\{(a,b)\in A \times B, , k \in \mathcal{Z}\: : \: ab = 2k\}\)\\
iv) \(\{(a,b)\in A \times B\: : \: a + b > 6\}\)\\
19. Hacer por extension, clasificar.\\
20. Reflex?  No. Sim? Si. As? No se. Trans? Si\\
21. Sea A ... \\
i) 4 ii) 1 iii) 5 iv) 6 v) 5 vi) ??? (Revisar, hecho de cuaderno viejo)\\
22.i) R? Si S? No As? Si Tr? Si\\
ii) Es reflexiva? \((a,a) \in \mathbb{N} \times \mathbb{N} : \: a+a=2a\) es par. Si. \\
Es simetrica? \((a,b) \in \mathbb{N} \times \mathbb{N} : \: a+b\: es\:par \) \(\Rightarrow(b,a) \in \mathbb{N} \times \mathbb{N}: b+a\) es par. Si, por conmutatividad de la suma. \\
Es antisimetrica? No. i.e. (2,4) y (4,2)\\
Es Transitiva? No. i.e. (1,1) y (2,2) \\
24. \={a}=\(\{a,b,f\}\) \={c}=\(\{e,f\}\) \={d}=\(\{d\}\)\\
la partición asociada a \(\mathcal{R}\) : \(\{a,b,f\} \cap \{c,e\} \cap \{d\}\)\\
25. Tiene cuatro clases de equivalencia. Cada clase está representada en la partición \(\mathcal{R}\) como cada "elemento" de esta. (Adjuntar gráfico)\\
26. Para probar que una relación es de equivalencia, necesitamos saber si es simultaneamente reflexiva, simétrica y transitiva.\\
\colorbox{lightgray}{Reflexiva} \(A \mathcal{R}A \Leftrightarrow A \triangle A \cap \{1,2,3\}=\emptyset\) \\
\(A \triangle A =\emptyset \Rightarrow \emptyset \cap \{1,2,3\} = \emptyset\)\\
\colorbox{lightgray}{\textit{Simétrica}}  \(A \mathcal{R}B \Leftrightarrow B \mathcal{R} A\) \\
\(A \mathcal{R}B \Leftrightarrow A \triangle B \cap \{1,2,3\}=\emptyset\) \\
\(A \mathcal{R}B \Leftrightarrow B \triangle A \cap \{1,2,3\}=\emptyset\) (Esto se puede ver por simetría de la diferencia simétrica[\(\triangle\)])\\ 
\colorbox{lightgray}{\textit{Transitiva}} Se puede demostrar por tabla de verdad\\
Finalmente, el ejercicio nos pide decidir si la relación es antisimétrica:\\
\colorbox{lightgray}{\textit{Antisimétrica}}\\
ii)Encontrar la clase de equivalencia de \(A=\{1,2,3\}\)\\
27. Similarmente al ejercicio anterior, debemos ver que cumpla las tres condiciones ya nombradas para probar que es una relación de equivalencia. \\
\colorbox{lightgray}{Reflexiva} \(x \mathcal{R}x \Leftrightarrow x^2 - x^2 = 93x - 93x = 0 \) \\ 
\colorbox{lightgray}{Simétrica} \(x \mathcal{R}y \Rightarrow y \mathcal{R}x \) \\ 
\(y \mathcal{R}x \Leftrightarrow y^2 - x^2 = 93y - 93x \)\\
Y multiplicando por \(-1\):\\
\(y \mathcal{R}x \Leftrightarrow x^2 - y^2 = 93x - 93y \Rightarrow x \mathcal{R}y\)\\
Y por lo tanto es simétrica.\\
\colorbox{lightgray}{Transitiva} \(x \mathcal{R}y \land y\mathcal{R}z \Rightarrow x \mathcal{R}z\)\\
\(x \mathcal{R}y \Leftrightarrow x^2 - y^2 = 93x - 93y \)\\
\(y \mathcal{R}z \Leftrightarrow y^2 - z^2 = 93y - 93z \)\\
Sumando ambas ecuaciones:\\
\(x^2 - y^2 + y^2 - z^2 = 93x - 93y + 93y - 93z \)\\
\(\Leftrightarrow x^2 - z^2 = 93x - 93z \Rightarrow x \mathcal{R}z \)\\
Entonces la relación es reflexiva, simétrica y transitiva y por lo tanto podemos concluir que es de equivalencia.
Por último, el ejercicio nos pide decidir si es antisimétrica:\\
\colorbox{lightgray}{Antisimetrica} \(x \mathcal{R}y \land y\mathcal{R}x \Rightarrow x=y\)\\
Esto se puede demostrar como Falso con un simple contraejemplo.
Si tomamos \(92\) y \(1\) como \(x\) e \(y\) respectivamente, obtenemos:\\
\(92\: \mathcal{R}\:1 \Leftrightarrow 92^2 - 1^2 = 93(92) - 93\)\\
\(\Leftrightarrow 92^2 - 1^2 = 93(92) - 93\)\\
\(\Leftrightarrow 92^2 - 1^2 = 93(92-1)\)\\
\(\Leftrightarrow (92+1)(92-1) = 93\cdot 91\)\\
\(\Leftrightarrow 93\cdot 91\ = 93\cdot 91 \Rightarrow 93 = 91\) \underline{Absurdo!}\\
Entonces no es antisimétrica.\\
ii) Hallar la clase de equivalencia de cada \(x \in A\) \\
28. i) \(A \mathcal{R} B \Leftrightarrow \#X = \#Y\)\\
Entonces, como hay 10 elementos en A, es posible formar 10 clases de equivalencia distintas, cada uno correspondiendo al cardinal indicado. \\
\(\#\overline{\{1\}}, \#\overline{\{1,2\}}, ... \#\overline{\{1,2...10\}}\) (nota: los cardinales están de más, arreglar)\\
ii)Infinitas clases de equivalencia. \\
\(\#\bar{1}, \#\bar{2}, ... \#\bar{n}\)
\subsection{Funciones}
29. i) No pues \( (3,a) \in \mathcal{R} \land (3,d) \in \mathcal{R} \Leftrightarrow a=d \) Absurdo.\\
ii) No pues 5 no se relaciona con nadie.\\
iii) Si, pues todo los elementos del conjunto de partida se relacionan. \\
iv) \(A=\mathbb{N}, B=\mathbb{R}, \mathcal{R}=\{(a,b) \in \mathbb{N}x\mathbb{R} : a=2b-3\}\)\\
Si, pues todos los elementos de \(\mathbb{N}\) están relacionados con algún elemento de \(\mathbb{R}\) \\
Esto se puede ver como: \(a=2b-3 \Rightarrow \frac{a+3}{2} = b \in \mathbb{R}\)\\
Es un numero real pues \(a\in\mathbb{N}\)\\
v)\(A=\mathbb{R}, B=\mathbb{N}, \mathcal{R}=\{(a,b) \in \mathbb{R}x\mathbb{N} : a=2b-3\}\)\\
No, pues no todos los elementos de \(\mathbb{R}\) están relacionados con los elementos de \(\mathbb{N}\) (Está al revés, se puede ver fácilmente buscando un contraejemplo con la expresión anterior)\\
vi)\(A=\mathbb{R}, B=\mathbb{N}, \mathcal{R}=\{(a,b) \in \mathbb{R}x\mathbb{N} : a+b \:es\:divisible\:por\:5\}\)\\
Luego, \(a+b=5k\) con \(k\in\mathbb{Z}\)\\
30. i) \colorbox{lightgray}{Inyectividad} Asumimos que es inyectiva, y probamos por contraejemplo que no lo es:\\
$f(1)=f(-1) \Rightarrow 1 = -1 $ Absurdo! Entonces no es inyectiva.\\
\colorbox{lightgray}{Sobreyectiva} \(12x^2 + 5 = y \Rightarrow \sqrt[]{\frac{y+5}{12}} = x\) con $\frac{y+5}{12} \geqslant 0$\\
Entonces, restringiendo la imagen(?), es sobreyectiva con $ \mathbb{R} \rightarrow \mathbb{R_\geqslant} 5$ \\ 
\colorbox{lightgray}{Biyectiva} No es biyectiva ya que es sobreyectiva pero no inyectiva. \\
ii)\\
31. i) $f(g(n,m))=\frac{(n(m+1))^2}{2}$\\
Finalmente nos queda:\\
\[
  f\circ g(n,m) =
  \begin{cases}
                                   \frac{(n(m+1))^2}{2} & \text{si $n=6k$} \\
                                   3(n(m+1))+1 & \text{si $n \neq 6k$} \\
  \end{cases}
%
\]
Habiendo calculado esto, evaluando:\\
\(f \circ g (2,5) = 72\)\\
\(f \circ g (3,2) = 28\)\\
\(f \circ g (3,4) = 46\)\\
ii) \[
  f\circ g(n) =
  \begin{cases}
                                   n & \text{si $n \leq 7$} \\
                                   2\:\sqrt[•]{n} - 1 & \text{si $n > 7$} \\
  \end{cases}
%
\]
con n > 0.\\
Para el primer caso, $f\circ g(n) \neq 13$ pues $n\leq 7$\\
Para el segundo caso, habra que encontrar un $n$ tal que $f\circ g(n) = 13 = 2\:\sqrt[•]{n} - 1 \Leftrightarrow \frac{14}{2} = 7 = \sqrt[•]{n} \Leftrightarrow |n| = 7^2$
Entonces o bien $n$ es $(-7)^2\notin \mathbb{N}$ o bien $n$ es $7^2 \in \mathbb{N}$\\
Luego $f\circ g (49) = 13$. \\
Para $15$ lo mismo, solo que el valor será $64$.\\
32. i) $f \circ g (x) = 2(x+3)^2 - 18$\\
$g \circ f (x) = 2x^2 - 15$\\
ii) $f \circ g : f(4n)$
\[
  f\circ g(n) =
  \begin{cases}
                                   4n - 2 & \text{si $n = 4k$} \\
                                   4n + 1 & \text{si $n \neq 4k,$}\: k \in \mathbb{Z} \\
  \end{cases}
%
\]
$g \circ f:$ \\
\[
  g\circ f(n) =
  \begin{cases}
                                   4(n-2) & \text{si $n = 4k$} \\
                                   4(n+1) & \text{si $n \neq 4k,$}\: k \in \mathbb{Z}\\
                                   
  \end{cases}
%
\]
iii) Acá todo bien con fog pero no con gof. Fijate: \\
$f\circ g (n) = (\sqrt[•]{n} + 5, 3\:\sqrt[•]{n})$\\
PERO $g \circ f: \mathbb{R} \rightarrow \mathbb{R}\times\mathbb{R} \rightarrow \mathbb{N} \rightarrow \mathbb{R}$ Absurdo! \\
33. $f \circ g = Id_\mathbb{N}$
\[
  f(n) =
  \begin{cases}
                                   \frac{n}{2} & \text{si n par} \\
                                   n & \text{si n impar}\\
                                   
  \end{cases}
  \:\:\:\:g(n) = 2n
%
\]
$f \circ g = \frac{2n}{n} = n$
\[
  g \circ f(n) =
  \begin{cases}
                                   \frac{2n}{2} & \text{si n par$$} \\
                                   2n & \text{si n impar $$}\\
                                   
  \end{cases}
%
\]
Como podemos ver, en el caso impar nos devuelve un numero par, por lo que la distingue de $Id_\mathbb{N}$ y cumple la condición indicada. Qué tul?\\
34. i)fog es inyectiva... *completar*\\
35. i) Para probar si es una relación de equivalencia, debemos ver las tres condiciones habituales. Además, el ejercicio nos pide ver si es antisimétrica:\\
\colorbox{lightgray}{Reflexiva} Como $f$ es biyectiva y el $Codominio=Im(f)$ y es sobreyectiva:\\
$\{1,...,10\}\subseteq Im(f)$ por ejemplo: $f(1)=1$\\
$f\mathcal{R}f \Leftrightarrow \exists n \in \{1,...,10\} / f(n) = 1 y f(n) = 1 $\\
\colorbox{lightgray}{Simétrica} $f \mathcal{R} g \Rightarrow f \mathcal{R} f $\\
$ \exists n / f(n) = g(n) = 1 \Rightarrow g \mathcal{R} f \Rightarrow g \mathcal{R} f $ \\
\colorbox{lightgray}{Transitiva} $f \mathcal{R}g \land g\mathcal{R}h \Rightarrow f \mathcal{R}h$\\
$ \exists n_1 / f(n_1)=g(n_1)=1$\\
$ \exists n_2 / f(n_2)=h(n_2)=2$\\
Como $f(n_1)=g(n_2)=1=g(n_2)=h(n_2)$ entonces $f(n_1)=h(n_2)$ y $f\mathcal{R}h$. \\
$\therefore$ es una relación de equivalencia. \\
Por último, nos falta probar la antisimetría:\\
\colorbox{lightgray}{Antisimetría} No. Punto. \\
ii)\\
36.\\

\section{Numeros Naturales e Inducción}
\subsection{Sumatoria y Productoria}
1. i) \[a)\sum_{k=0}^{100} k\;\;\;\;b) \sum_{k=0}^{10} 2^k \;\;\;\;c)\sum_{k=0}^{11} (-1)^k k^2\;\;\;\;d) \sum_{k=0}^{10} (2k+1)^k \;\;\;\;e)\sum_{k=0}^{2n+1} 2k+1 \;\;\;\;f)\sum_{k=0}^{n} kn  \]
ii) \[a)\prod{k=5}^{100} k\;\;\;\;b) \prod_{k=0}^{10} 2^k \;\;\;\;c)\prod_{k=1}^{n} kn \]
2. i) $2+4+...+2(n-5)+2(n-4)$\\
ii)$\frac{1}{2}+\frac{1}{6}+...+\frac{1}{n(n+1}+\frac{1}{(n+1)(n+2)}+...+\frac{1}{(2n-1)(2n)}+\frac{1}{(2n)(2n+1)}$\\
iii) $\frac{n+1}{2}+\frac{n+2}{4}+...+\frac{2n}{2n}+\frac{2n+1}{2n}??$\\
iv) $\frac{n}{1}+\frac{1}{2}+...+\frac{n}{n^2}+\frac{n}{(n+1)^2}$\\
v) copiar despues\\

3. Calcular.\\
a) 2n(n+1)+n\\
b) ...\\
4.\\
\subsection{Inducción}

5. i) El cuadrado es de 7x7, luego $7^2$ lo que coincide con:
\[\text{} \sum_{i=1}^{7} (2i-1) = 49 \]
Entonces, el caso para n cuadrados:
\[\text{} \sum_{i=1}^{n} (2i-1) = n^2 \]

ii) Con suma de Gauss:
\[\sum_{i=1}^{n} (2i-1) =  2\sum_{i=1}^{n} i -  \sum_{i=1}^{n} n = \frac{2(n(n+1))}{2} - n = n^2+n-n = n^2\]
iii) Con inducción:
\[\text{Caso base (n=1)} \sum_{i=1}^{1} (2i-1) = 1 \]
\[\text{Paso Inductivo}\;\;\;\; HI: \sum_{i=1}^{k} (2i-1) = k^2\;\;\;\; QVQ: \sum_{i=1}^{k+1} (2i-1) = (k+1)^2 \]
\[\sum_{i=1}^{k+1} (2i-1) = \sum_{i=1}^{k} (2i-1) + 2k + 1 \stackrel{HI}{=} k^2 + 2k + 1 = (k+1)^2 \]
\hfill$\square $\\
6.Completar\\
7.Completar\\
8. \underline{Caso Base}\\
\[ P(1) = a^1 - b^1 = (a-b)\sum_{i=1}^{1} a^{i-1}b^{1-i} = a-b \]
\underline{Paso Inductivo}\\
\[P(k) V \Rightarrow P(k+1) V\]
Hipótesis Inductiva(HI):\[ P(k) = a^k - b^k = (a-b)\sum_{i=1}^{k} a^{i-1}b^{k-i} \]
Quiero ver que(QVQ):\[ P(k+1) = a^{k+1} - b^{k+1} = (a-b)\sum_{i=1}^{k+1} a^{i-1}b^{k+1-i} \]

Reescribimos:\\
HI: \[ P(k) = \frac{a^{k} - b^{k}}{a-b} = \sum_{i=1}^{k} a^{i-1}b^{k-i} \]


\begin{equation} 
\begin{split}
P(k+1) & = \frac{a^{k+1} - b^{k+1}}{a-b} = \sum_{i=1}^{k+1}a^{i-1}b^{k+1-i} \\
 &  = \sum_{i=1}^{k}a^{i-1}b^{k-i}b+a^{(k+1)-1}b^{k+1-(k+1)} \\
 & \stackrel{HI}{=} b\sum_{i=1}^{k}a^{i-1}b^{k-i}+a^{k} \\
 & = b \frac{a^{k} - b^{k}}{a-b} +a^{k} \\
 & = \frac{a^{k}b - b^{k+1}}{a-b} +\frac{a^{k}(a-b)}{a-b} = \frac{a^{k}b - b^{k+1}+a^{k+1}-a^{k}b}{a-b} \\
 & = \frac{a^{k+1}- b^{k+1}}{a-b}
\end{split}
\end{equation}
\hfill$\square $\\
Cómo queríamos probar. Para la geométrica se puede plantear [...], luego:\\
 \[ \frac{a^{n} - 1}{a-1} = \sum_{i=1}^{n} a^{i-1} \]
9.i)\underline{Caso Base}\\
\[ P(1) = \sum_{i=1}^{1} a_{i+1}-a_{i} = a_2 - a_1 \]
\underline{Paso Inductivo}\\
\[P(k) V \Rightarrow P(k+1) V\]
Hipótesis Inductiva(HI):\[ P(k) = \sum_{i=1}^{k} a_{i+1}-a_{i} = a_{k+1} - a_1 \]
Quiero ver que(QVQ):\[ P(k+1) = \sum_{i=1}^{k+1} a_{i+1}-a_{i} = a_{k+2} - a_1  \]


\begin{equation}
\begin{split}
P(k+1) & = \sum_{i=1}^{k+1} a_{i+1}-a_{i} = a_{k+2} - a_1 \\
 & = \sum_{i=1}^{k} a_{i+1}-a_{i} + a_{k+2} - a_{k+1}\\
 & \stackrel{HI}{=} a_{k+1} - a_1 + a_{k+2} - a_{k+1} \\
 & = a_{k+2} - a_1
\end{split}
\end{equation}
\hfill$\square $\\
Cómo queríamos probar.\\

ii)Acá tenemos que conjeturar en base a lo que nos dan, ya que no tenemos una formula cerrada. Una vez que la encontremos, probamos por inducción nuestra conjetura.\\
"Conjeturo":\\
\begin{align*}
n=1           &           &  & \frac{1}{2}\\
n=2           &     	  &  \frac{1}{2}+\frac{1}{6}&=\frac{2}{3}\\
n=3           &           &  \frac{2}{3}+\frac{1}{12}&=\frac{3}{4}
\end{align*}
Esto me suena de la forma $\frac{n}{n+1}$ Lo pruebo por inducción:
\[ P(n) = \sum_{i=1}^{n} \frac{1}{i(i+1)} = \frac{n}{n+1} \]

\underline{Caso Base}\\
\[ P(1) = \sum_{i=1}^{1} \frac{1}{i(i+1)} = \frac{1}{1+1} = \frac{1}{2} \]
\underline{Paso Inductivo}\\
\[P(k) V \Rightarrow P(k+1) V\]
Hipótesis Inductiva(HI):\[ P(k) = \sum_{i=1}^{k} \frac{1}{i(i+1)} = \frac{k}{k+1} \]
Quiero ver que(QVQ):\[ P(k+1) = \sum_{i=1}^{k+1} \frac{1}{i(i+1)} = \frac{k+1}{k+2} \]
\begin{equation}
\begin{split}
P(k+1) & = \sum_{i=1}^{k+1} \frac{1}{i(i+1)}  \\
 & = \sum_{i=1}^{k} \frac{1}{i(i+1)} + \frac{1}{(k+1)(k+2)} \\
 & \stackrel{HI}{=} \frac{1}{k+1} + \frac{1}{(k+1)(k+2)} \\
 & = \frac{k(k+2)+1}{(k+1)(k+2)} = \frac{(k+1)^2}{(k+1)(k+2)} = \frac{(k+1)}{(k+2)} 
\end{split}
\end{equation}
\hfill$\square $\\
Cómo queríamos probar.\\

iii) Por la sugerencia, observemos que:
\[\sum_{i=1}^{n} \frac{1}{(2i-1)(2i+1)} = \frac{1}{2} \sum_{i=1}^{n} \frac{2}{(2i-1)(2i+1)} \]
Entonces, conjeturando solo la sumatoria del lado derecho de la igualdad:\\
\begin{align*}
n=1           &           &  & \frac{2}{3}\\
n=2           &     	  &  \frac{2}{3}+\frac{2}{15}&=\frac{4}{5}\\
n=3           &           &  \frac{4}{5}+\frac{2}{35}&=\frac{6}{7}
\end{align*}
Luego:
\[\sum_{i=1}^{n} \frac{1}{(2i-1)(2i+1)} = \frac{1}{2}  \frac{2n}{2n+1} = \frac{n}{2n+1}  \]

Pareciera ser $\frac{n}{2n+1}$ para $n \in \mathbb{N}$. Pruebo por Inducción:\\
\[ P(n) = \sum_{i=1}^{n} \frac{1}{(2i-1)(2i+1)} = \frac{n}{2n+1} \]
\underline{Caso Base}\\
\[ P(1) = \sum_{i=1}^{1} \frac{1}{(2i-1)(2i+1)} = \frac{1}{3} \]
\underline{Paso Inductivo}\\
\[P(k) V \Rightarrow P(k+1) V\]
Hipótesis Inductiva(HI):\[ P(k) = \sum_{i=1}^{k} \frac{1}{(2i-1)(2i+1)} = \frac{k}{2k+1} \]
Quiero ver que(QVQ):\[ P(k+1) = \sum_{i=1}^{k+1} \frac{1}{(2i-1)(2i+1)} = \frac{k+1}{2k+3} \]

\begin{equation}
\begin{split}
 P(k+1) & = \sum_{i=1}^{k+1} \frac{1}{(2i-1)(2i+1)}  \\
 & = \sum_{i=1}^{k} \frac{1}{(2i-1)(2i+1)} + \frac{1}{(2(k+1)-1)(2(k+1)+1)} \\
 & \stackrel{HI}{=} \frac{k}{2k+1} +  \frac{1}{(2k+1)(2k+3)} \\
 & = \frac{k(2k+3)+1}{(2k+1)(2k+3)} = \frac{2k^2+3k+1}{(2k+1)(2k+3)}\\
 & = \frac{2k^2+3k+1}{4k^2+8k+3} = \frac{2k^2+3k+1}{2(8k^2+4k+\frac{3}{2}} \\
\end{split}
\end{equation}
Factorizando:
\begin{equation}
\begin{split}
 & = \frac{(k+\frac{1}{2})(k+1)}{2(k+\frac{1}{2})(k+\frac{3}{2})} \\
 & = \frac{(k+1)}{2(k+\frac{3}{2})} = \frac{(k+1)}{2k+3}
\end{split}
\end{equation}
\hfill$\square $\\
Cómo queríamos probar.\\
10.Completar. Checkear iv-vii.\\
11. Prestale atención a este ejercicio porque es GOOOOD. \\
i) Fijate que nos piden probar que $(a_n)_{n\in\mathbb{N}}$ es una sucesión de numeros reales, todos \textbf{del mismo signo!!} Y que es mayor estricta que -1. Entonces, eso nos deja dos casos posibles:\\

\[a_n > -1, todos\:mismo\:signo 
\]
$$
[ a_n > -1 ]
\begin{cases}
(-1, 0) \\
(0, +\infty)
\end{cases}
$$

Planteemos el caso base y paso inductivo. Defino el predicado:\\
\[ P(n) = \prod_{i=1}^{n}(1+a_i) \geq 1 + \sum_{i=1}^{n} a_i \]
\underline{Caso Base}\\
\[ P(1) = \prod_{i=1}^{1}(1+a_i) \geq 1 + \sum_{i=1}^{1} a_i \Leftrightarrow 1 + a_1 \geq 1 + a_1 \]
\underline{Paso Inductivo}\\
\[P(k) V \Rightarrow P(k+1) V\]
Hipótesis Inductiva(HI):\[ P(k) = \prod_{i=1}^{k}(1+a_i) \geq 1 + \sum_{i=1}^{k} a_i \]
Quiero ver que(QVQ):\[P(k+1) = \colorbox{Cerulean}{$\prod\limits_{i=1}^{k+1}(1+a_i)$} \geq   \colorbox{Apricot}{$1 + \sum\limits_{i=1}^{k+1} a_i $}  \]
Acá prestale atención a los colorcitos, porque vamos a "separar" ambas partes de la ecuacion para despues manipularlas en nuestra prueba:\\

\[ \colorbox{Cerulean}{$\prod\limits_{i=1}^{k+1}(1+a_i)$} = (1+a_{k+1}) \prod\limits_{i=1}^{k}(1+a_i) \stackrel{HI}{\geq} \colorbox{LimeGreen}{$(1 + \sum\limits_{i=1}^{k} a_i)(1+a_{k+1})$} \]

\[ \colorbox{Apricot}{$1 + \sum\limits_{i=1}^{k+1} a_i $} = 1+a_{k+1} + \sum\limits_{i=1}^{k} a_i \]
\\
Entonces tenemos que:
\[\colorbox{LimeGreen}{$ (1 + \sum\limits_{i=1}^{k} a_i)(1+a_{k+1})$} = 1 + a_{k+1} + \sum\limits_{i=1}^{k} a_i + a_{k+1} \sum\limits_{i=1}^{k} a_i \geq \colorbox{Apricot}{$1 + \sum\limits_{i=1}^{k+1} a_i$} = 1 + \sum\limits_{i=1}^{k} a_i + a_{k+1} \] 
Elminando términos de la inecuación, nos queda:\\
\[ a_{k+1} \sum\limits_{i=1}^{k} a_i \geq 0\] 

y como tenemos una multiplicación de dos elementos, y por enunciado, \textbf{todos} los elementos son del \textbf{mismo signo}, entonces o bien $a_{k+1} \land \sum\limits_{i=1}^{k} a_i \geq 0 $ es mayor a cero, o bien $a_k+1 \land \sum\limits_{i=1}^{k} a_i \leq 0 $  lo que cumple que $a_{k+1} \sum\limits_{i=1}^{k} a_i \geq 0$ ya que multiplicar dos elementos con un mismo signo nos da un número positivo.
 
Entonces cómo probamos que p(k) Verdadero entonces p(k+1) Verdadero, y p(1) Verdadero (nuestro Caso Base) entonces p(n) es Verdadero $\forall n\in\mathbb{N}$\\

\hfill$\square $\\
ii.\\
12. Probar que:\\
i. $n! \geq 3^{n-1}, \forall n\geq 5 $\\

Estamos con un típico ejercicio de inducción corrida. Fijate que arranca para todo n mayor o igual a 5, por lo tanto nuestro caso base lo tendremos que probar a partir de ese número.
Lo pruebo por inducción:\\
\[ P(n) = n! \geq 3^{n-1} \]
\underline{Caso Base}\\
\[ P(5) = 5! \geq 3^{4} \] 
\underline{Paso Inductivo}\\
\[P(k) V \Rightarrow P(k+1) V\]
Hipótesis Inductiva(HI):\[ P(k) = k! \geq 3^{k-1} \]
Quiero ver que(QVQ):\[ P(k+1) = (k+1)! \geq 3^{k} \]

\begin{align*}
\begin{split}
 (k+1)! = k!(k+1) \geq 3^{k-1}(k+1) & \geq 3^k  \\
 k+1 & \geq 3^{k-(k-1)} \\
 k & \geq 3 - 1 \\
 k & \geq 2\\
\end{split}
\end{align*}
y como $k \geq 5$ esto es verdadero. Por lo tanto, nuestro paso inductivo es verdadero, y como el caso base es verdadero, entonces $p(n)$ verdadero $\forall n \in \mathbb{N}$.
\hfill$\square$\\
ii. $3^n - 2^n > n^3, \forall n \geq 4$\\
Lo pruebo por inducción:\\
\[ P(n) = 3^n - 2^n > n^3 \]
\underline{Caso Base}\\
\[ P(4) = 3^4 - 2^4 > 4^3 \] 
\underline{Paso Inductivo}\\
\[P(k) V \Rightarrow P(k+1) V\]
Hipótesis Inductiva(HI):\[ P(k) = 3^k - 2^k > k^3 \]
Quiero ver que(QVQ):\[ P(k+1) = 3^{k+1} - 2^{k+1} > (k+1)^3 \]
Recordemos que: $ a>b \Leftrightarrow \colorbox{LimeGreen}{x} + a > \colorbox{LimeGreen}{x} + b  $\\
\begin{align*}
\begin{split}
\colorbox{TealBlue}{$3^{k+1} - 2^{k+1}$} = 3\cdot 3^{k} - 2\cdot 2^{k} = \colorbox{LimeGreen}{$2\cdot 3^{k} -  2^{k}$} + \colorbox{Apricot}{$3^k - 2^k$} \stackrel{\colorbox{Apricot}{HI}}{>} \colorbox{LimeGreen}{$2\cdot 3^{k} -  2^{k}$} + \colorbox{Apricot}{$k^3$} & \stackrel{\colorbox{TealBlue}{{QVQ}}}{>} \colorbox{TealBlue}{$(k+1)^3$} \\
 \textit{Y como arranca desde $4 \geq 0$ va a estar acotado por  } 2\cdot 3^4 - 2^4 + 4^3 & > 5^3\\
 \textit{Tenemos que: } \underset{\geq 162}{2\cdot 3^{k} -  2^{k}} + \underset{\geq 64}{k^3} & >\underset{\geq 125}{(k+1)^3}\\
\end{split}
\end{align*}
y como $k \geq 4$ esto es verdadero. Por lo tanto, nuestro paso inductivo es verdadero, y como el caso base es verdadero, entonces $p(n)$ verdadero $\forall n \in \mathbb{N}$.\\
.$\hfill \square$\\
iii. $\sum\limits_{i=1}^{n} \frac{3^i}{i!} < 6n-5, \forall n \geq 3 $
Lo pruebo por inducción:\\
\[ P(n) =  \sum\limits_{i=1}^{n} \frac{3^i}{i!} < 6n-5, \forall n \geq 3 \]
\underline{Caso Base}\\
\[ P(3) = \sum\limits_{i=1}^{3} \frac{3^i}{i!} = 3 + \frac{3^2}{2!} + \frac{3^3}{3!} < 6(3)-5 \Leftrightarrow 12 < 13 \] 
\underline{Paso Inductivo}\\
\[P(k) V \Rightarrow P(k+1) V\]
Hipótesis Inductiva(HI):\[ P(k) =  \sum\limits_{i=1}^{k} \frac{3^i}{i!} < 6k-5 \]
Quiero ver que(QVQ):\[ P(k+1) =  \sum\limits_{i=1}^{k+1} \frac{3^i}{i!} < 6k + 1\]

\[ P(k+1) =  \sum\limits_{i=1}^{k+1} \frac{3^i}{i!} = \sum\limits_{i=1}^{k} \frac{3^i}{i!} + \frac{3^{k+1}}{(k+1)!} < 6k - 5 + \frac{3^{k+1}}{(k+1)!} < 6k+1 \]

Como $k \geq 3$ entonces cumple la desigualdad por acotacion. Luego, por ppio. de induccion corrido, $p(3)$ verdadero y $p(k) V \Rightarrow P(k+1)V$ entonces $p(n)$ verdadero $\forall n \in \mathbb{N}$
\hfill$\square$\\
13. El Caso Base arranca en 5. Probemos que $n \geq 5$ por induccion.
\[ P(n) =  n^2 + 1 < 2^n \]
\underline{Caso Base}\\
\[ P(5) = 5^2 + 1 < 2^5 \;\;\textit{Verdadero.} \] 
\underline{Paso Inductivo}\\
\[P(k) V \Rightarrow P(k+1) V\]
Hipótesis Inductiva(HI):\[ P(k) =  k^2 + 1 < 2^k \]
Quiero ver que(QVQ):\[ P(k+1) =  (k+1)^2 + 1 < 2^{k+1} \]
Como vimos anteriormente que $n^2 +1 < 2^n$ es equivalente a $2^n > n^2 +1$, podemos expresar el paso inductivo como:\\
\begin{align*}
\begin{split}
2^{k+1}=2 \cdot 2^k \overset{HI}{>} 2(k^2+1) & \overset{qvq}{>} (k+1)^2 +1\\
2k^2 + 1 & > (k+1)^2 \\
k^2 & > 2k \\
\underset{\geq 25}{k^2} - \underset{\geq 10}{2k} & > 0 \\
\textit{pues $k \geq 5$ entonces es verdadero}
\end{split}
\end{align*}
Por lo tanto, por principio de inducción corrida, cómo nuestro caso base $p(5)\;\;verdadero$  y $p(k) \;\;verdadero \Rightarrow p(k+1)\;\;verdadero$ luego $p(n)\;\;verdadero, \forall n \in \mathbb{N}_{\geq 5}$
\hfill$\square$\\
14.
\end{document}
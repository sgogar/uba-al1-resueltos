\documentclass[10pt, a4paper]{book}
\usepackage[utf8]{inputenc}
\usepackage{setspace}
\usepackage{parskip}
\usepackage{amsfonts}
\title{Algebra I - Resueltos 2021}
\author{Por Libatio\thanks{Agradecimientos y coautoria de tew, el sinse, ido, luchando por la vida}
}
\doublespacing
\begin{document}
\maketitle
Esto es una beta con los ejercicios resueltos de Algebra hasta el primer parcial. Puede contener \textbf{errores}, así como también aciertos. Lo voy a terminar post 15 de Diciembre cuando me libere con los parciales. 

Muchas gracias a Teresa, Nico, Georgi, Vicky, Juli y Sergio por hacer esto posible [etc]
\chapter{Conjuntos, Relaciones y Funciones}

\section{Conjuntos}
1. i) Verdadero. ii) Verdadero. iii) Verdadero. iv) Falso. v) Falso.\\
2. i) Falso pues \(3\not\in A\). ii) Verdadero. iii) Falso pues \(\lbrace\lbrace3\rbrace\rbrace\subset A\).\\
 iv) Verdadero. v) Verdadero. vi) Verdadero. vii) Verdadero \\
 viii) Falso pues \(3\notin A\). ix) Falso pues \(\emptyset \not\in A\) \\
 x) Verdadero. xi) Falso pues \(A \not\in A\) xii) Verdadero. \\
3.
i) Como B tiene a los elementos de A uno a uno (recordemos que no importa el orden) entonces \(A \subseteq B\)\\
ii) \(A \not \subseteq B\), pues \(\not\exists x \in A\) tal que \(x \not\in B (i.e. ./x = 3)\)\\
iii) \(A \not \subseteq B\), pues \(\sqrt[]{x^2}<\sqrt[]{3}\leftrightarrow |x|<\sqrt[]{3} \) y como \(\sqrt[]{3}<3 \ luego \exists x\in A / x \not\in B\)\\
iv) \(A \not \subseteq B\)\\
4. \[
A=\{1, -2, 7, 3\} , B=\{1, \{3\}, 10\} C=\{-2, \{1,2,3\}, 3\}
\]
\[i) A \cap (B \triangle C)\]
\[(B \triangle C) = \{1,\{3\}, 10, -2, \{1,2,3\},3\} \:y\: A=\{1, -2, 7, 3\}\:entonces: \]
\[A \cap (B \triangle C) = \{1, -2, 3\}\]
\[ii) (A \cap B) \triangle (A \cap C)\]
\[(A \cap B)=\{1\}\:y\:(A \cap C)=\{-2,3\}\]
\[(A \cap B) \triangle (A \cap C)=\{1,-2,3\}\]
\[iii) A^{c}\cap B^{c}\cap C^{c}\:\:bueno \:hacelo\: vos.\]
asdasdasd\\
5. Dados \(A\;B\;C\) subconjuntos de un conjunto ref V, describir \((A\cap B\cap C)^{c}\) en terminos de\\
\textit{Intersecciones y complentos:}\\
\(((A\cap B)\cap C)^{c}\:\:\:\:\:\:\:\:Por\:asociatividad\: de\: la\: union\)\\
\((A\cap B)^{c}\cap C^{c}\:\:\:\:\:\:\:\:\:De\: Morgan\)\\
\(A^{c}\cap B^{c}\cap C^{c}\:\:\:\:\:\:\:\:Idem\)\\
\textit{Uniones y complementos:}\\
\(((A\cap B)\cap C)^{c}\:\:\:\:\:\:\:\:Por\:asociatividad\: de\: la\: union\)\\
\((A\cap B)^{c}\cup C^{c}\:\:\:\:\:\:\:\:\:De\: Morgan\)\\
\(A^{c}\cup B^{c}\cup C^{c}\:\:\:\:\:\:\:\:Idem\)\\
6. Hecho en dibujitos que hay que subir (y me da pajulis)\\
7. i) \((A\cap B^{c})\cup((B\cap C) \cup A^{c})\)\\
ii) \((A\triangle C)\cap B^{c}\)\\
iii) \((A\cap B)\triangle(B\cap C)\cup(A\cap C)\cap B^{c}\)\\
8. i) \(A={1},\:\:\:\mathcal{P}(A)=\{\{1\},\emptyset\}\)\\
ii)\(A=\{a,b\},\:\:\:\mathcal{P}(A)=\{\{a\},\{b\},\{a,b\},\emptyset\}\)\\
iii)\\
9. Probar etc. \\
Bueno, qué corno hago? En criollo, tenemos que probar que partes de A es un subconjunto de partes de B \textbf{\textit{sí sólo sí}} A está contenido en B. Para probar este tipo de implicaciones, tenemos que probar la ida (\(\Rightarrow\)) y luego la vuelta (\(\Leftarrow\))\\
Recuerdo: \(\mathcal{P}(A)=\{X:X\subseteq A\}\:\: o\:\:tambien\:\: X \in \mathcal{P}(A)\iff X\subseteq A\)\\
Ida (\(\Rightarrow\)): Asumo que \(\mathcal{P}(A)\subseteq \mathcal{P}(B)\) quiero ver que \(A\subseteq B \)
Tomo \(x \in A \) quiero ver que \(x \in B \). Se que \(A \subseteq A\) . A, el conjunto, es un elemento de \(\mathcal{P}(A)\) y se escribe \(A\in \mathcal{P}(A)\) y como se que \(\mathcal{P}(A)\subseteq \mathcal{P}(B)\), entonces por transitividad,  \(A\in \mathcal{P}(B)\). Pero, ¿qué significa que \(A\in \mathcal{P}(B)\)? Por definición de \(\mathcal{P}(B)\), A tiene que ser un subconjunto de B. Y como es un subconjunto,  \(A\subseteq B \).\\
Vuelta (\(\Leftarrow\)): Sabiendo que \(A\subseteq B \) q.v.q. \(\mathcal{P}(A)\subseteq \mathcal{P}(B)\) \\
Tomo \(X \in \mathcal{P}(A) \) entonces \(X \subseteq A \) y entonces \(X \subseteq A \subseteq B\) y por transitividad, \(X \subseteq B\). A la vez, por definici'on, \(X \subseteq B \subseteq \mathcal{P}(B) \), luego \(X \subseteq {P}(B) \), por lo tanto \(X \subseteq \underline{\mathcal{P}(A) \subseteq \mathcal{P}(B)} \) que es lo que queriamos probar.\\
10. Tablas de verdad\\
11. Tablas de verdad \\
12. Tablas de verdad (Agregar notas sobre cuantificadores) \\
13. Tablas de verdad \\
14. Similar \\
15. Sean \(A=\{1,2,3\} B=\{1,3,5,7\}\). Hallar \(A\times A, A\times B, (A\cap B)\times(A\cup B)\)\\
\(A \times A= \{(1,1),(1,2),(1,3),(2,1),(2,2),(2,3),(3,1),(3,2),(3,3)\} \)\\
\(A \times B= \{(1,1),(1,3),(1,5),(1,7),(2,1),(2,2),(2,5),(2,7),(3,1),(3,3),(3,5),(3,7)\} \)\\
\((A\cap B)\times(A\cup B)=\{(1,1),(1,3),(1,5),(1,7),(3,1),(3,3),(3,5),(3,7)\} \)\\
16. Probalo vos. \\
\section{Relaciones}
17. i. Es relacion? Si, pues todos los elementos que se relacionan de A en B existen en A o B segun especifica la relacion.\\
ii. No es, ya que \( (3,2) \in R\) pero \(2 \not\in B\)\\
iii. Si, idem i.\\
iv. Si, idem i. \\
18. i) \(\{(a,b)\in A \times B\: : \: a\leq b\}\)\\
ii) \(\{(a,b)\in A \times B\: : \: a > b\}\)\\
iii) \(\{(a,b)\in A \times B, , k \in \mathcal{Z}\: : \: ab = 2k\}\)\\
iv) \(\{(a,b)\in A \times B\: : \: a + b > 6\}\)\\
19. Hacer por extension, clasificar.\\
20. Reflex?  No. Sim? Si. As? No se. Trans? Si\\
21. Sea A ... \\
i) 4 ii) 1 iii) 5 iv) 6 v) 5 vi) ??? (Revisar, hecho de cuaderno viejo)\\
22.i) R? Si S? No As? Si Tr? Si\\
ii) Es reflexiva? \((a,a) \in \mathbb{N} \times \mathbb{N} : \: a+a=2a\) es par. Si. \\
Es simetrica? \((a,b) \in \mathbb{N} \times \mathbb{N} : \: a+b\: es\:par \) \(\Rightarrow(b,a) \in \mathbb{N} \times \mathbb{N}: b+a\) es par. Si, por conmutatividad de la suma. \\
Es antisimetrica? No. i.e. (2,4) y (4,2)\\
Es Transitiva? No. i.e. (1,1) y (2,2) 

\end{document}
